\documentclass{article}

\usepackage{verbatim}
\usepackage{amsthm}
\usepackage{amsfonts}

\newcommand{\Code}[1]{\texttt{#1}}
\newcommand{\Config}[2]{\ensuremath{\langle #1, #2 \rangle}}
\newcommand{\Rule}[2]{\ensuremath{#1 \hookrightarrow #2}}
\newcommand{\Trans}[3]{\ensuremath{#1 \rightarrow_{#2} #3}}

\newcommand{\powerset}[1]{P(#1)}
\newcommand{\epspath}[3]{\ensuremath{#1 to #2 via #3}}
\newcommand{\subsubsubsection}[1]{\textbf{#1}}
\newcommand{\bbN}{\mathbb{N}}
\newcommand{\meet}{\sqcap}
\newcommand{\extend}{\otimes}

\newtheorem{definition}{Definition}

\begin{document}

This document will work up to a description of WPDSs as they are used
in program analysis. We will start with unweighted pushdown systems,
discussing their applications primarily in terms of verifying Boolean
programs, then move on to the weighted versions. In order to set the
stage for describing unweighted PDSs, we start with describing boolean
programs, and use that as a launching point for discussing trasition
systems, which PDSs define.

\section{Boolean Programs}
The definition of a Boolean program is very simple: a Boolean program
is just a program where every variable has type Bool.

We will defer the question of why Boolean programs are useful for a
moment, and instead concentrate on their semantics. I'm going to leave
details of how the semantics are specified (e.g. small/large-step
operational, denotational, etc.) aside. What we're really interested
in at the moment is the structure of states and which states are
successors of which other states, but how you actually
\emph{determine} the successor relation is less relevant.

One other important difference is that Boolean programs are allowed to
have nondeterministic branches, notated as \Code{if(*)}. In terms of
reachability, both branches of such a branch are reachable; this means
that if you are trying to determine whether an error point is
reachable, it can be located in either branch (or rely on variable
settings from either branch). The same can apply to loops.

Boolean programs can also have an \Code{assert} statement, which marks
an error line. Assertions don't have attached conditions, but such
conditions can of course be simulated as \Code{if(...) assert;}.

The initial valuation of globals is undefined, as are the initial
values of non-parameter locals when a procedure is called. The program
nondeterministically sets them. (You can think of uninitialized
variables as one way of a program receiving input.)

At any given point, the state of a Boolean program consists of a
valuation of the local variables along with a stack of activation
records. Each activation record holds a valuation of the locals of the
corresponding procedure as well as the address of the next statement
to be executed. (For us, ``address'' means ``line number''; it's more
abstract than the actual CPU's instruction pointer.)

| There are similarities and differences between this presentation and
| what you see with ``normal'' programs. There are two main
| differences. First, we've dropped the possibility of a heap (with
| only Boolean values we have no pointers). Second, the statement
| addresses we store in the stack are offset from what is
| typical. Usually the current address is stored in a register, which
| is sort of a global location; our semantics will store the current
| address in the current stack frame. Furthermore, if \Code{foo()}
| calls \Code{bar()}, then the address in \Code{foo()} that execution
| should return to when \Code{bar()} returns will be stored in
| \Code{bar()}'s activation record -- our semantics will store that
| address in \Code{foo()}'s activation record.

Consider the following example, program 1:
[TODO: insert]

We will write a configuration as follows:
\Config{g=T}{\frame{1}{h=T}}; this is a possible starting
configuration of program 1. The first element in the outer tuple is
the global state, and the second element is the stack. The stack is
written as a sequence of pairs, the first element of which is the
statement that is \emph{about to execute} when control returns to that
procedure (for the topmost element, that is ``now.'') , and the second
element of which is a valuation of the locals. The top of the stack is
give first. 

Here is another configuration that arises later in the execution of
program 1, starting with g=T and h=T:

\begin{verbatim}
                 AR of f (top of stk)  AR of main
                 -------------------   ------------
   \Config{g=T}{\frame{7}{a1=T, a2=F} \frame{3}{h=T}}
           ^           ^         ^           ^
        val of     next line    val of      next line to exec
        global     to exec.     locals      after f() returns
\end{verbatim}


If the program can start in state $s_1$, execute the statement named
by the topmost stack frame in $s_1$, and go to state $s_2$, then we
will write $s_1 \rightarrow s_2$. Most states will have only one
successor, but the head of a nondeterministic branch or loop will have
two, and when calling a procedure with $n$ non-parameter locals, there
will be $2^n$ possible successors (one for each possible valuation of
that procedure's locals).

We can picture the possible executions of the example program in the
following figure:

  T 0T               T 0F            F 0T          F 0F
  T 1T               T 1F            F 1T          F 1F
  T 5TT 1rT          T 5TF 1rF       F 5FT 1rT     F 5FF 1rF
  T 6TT 1rT          T 6TF 1rF       F 6FT 1rT     F 6FF 1rF
  T 7TT 1rT          T 7TF 1rF       F 8FT 1rT     F 8FF 1rF
  T 5TT 7rTT 1rT     T 5FT 7rTF 1rF  T 9FT 1rT     F 9FF 1rF
  T 6TT 7rTT 1rT     T 6FT 7rTF 1rF  T 1rT         F 1rF
  T 7TT 7rTT 1rT     T 8FT 7rTF 1rF  T 2rT         F 2F
  T 5TT 7rTT 7rTT... T 9FT 7rTF 1rF  T 3rT         F 4F
       ...           T 7rTF 1rF      assert        halt
                     T 9TF 1rF
                     T 1rF
                     T 2F
                     T 3F
                     assert


Any path in this graph corresponds to a possible execution of the
program. Note that there are three possibile behaviors: the program
can not terminate (column 1), can terminate execution in an error
(columns 2 and 3), or can complete successfully (column 4).

Our goal will be to determine whether it is possible for a Boolean
program to reach an assertion. With the example program, it is, but if
if \Code{h} and \Code{g} were initialized to have the same value, it
would not be possible. (Detecting nontermination is a harder problem,
which we will not address.)

-----------------
Why are Boolean programs interesting?

First of all, Boolean programs can act as a relatively simple
abstraction of a program that you wish to analyze. Variables in the
Boolean program need not correspond exactly to variables in the
original program; instead, they can represent \emph{predicates}. For
instance, a Boolean variable \Code{b} could represent the predicate
\Code{x==y} (where \Code{x} and \Code{y} are variables in the original
program).

However, the real benefit comes when combining a tool that analyzes
Boolean programs into a larger tool that performs somtehing called
\emph{counterexample-guided abstraction refinement} (CEGAR). The
overall method can be diagrammed as follows:

[CEGAR picture]

Basically what happens is the tool takes the original program and
produces a very coarse abstraction as a Boolean program. It then
checks that initial abstraction to figure out if there is an error in
it, using the process we are about to describe. If not, because all
Boolean program abstractions are sound, the original program is proved
to be free of that kind of error.

Conversely, suppose that the Boolean program has an error. The path
through the Boolean program is passed to another tool which looks to
see whether that path is feasible in the original program or whether
it is an imprecision introduced by the abstraction. If the path is
actually feasible, it means we've found a bug in the original program,
and our analysis tool is complete.

The magic of the CEGAR loop happens in the remaining case: there is a
path to an assertion in the Boolean program abstraction but that path
is an imprecision and doesn't correspond to something actually
executable in the original program. In this case, the tool doesn't
just say ``this path is not feasible'' but returns an explanation of
\emph{why}. That \emph{why} is turned into some additional predicates
that the Boolean program should track, which will increase the
precision. The Boolean program abstraction is \emph{refined} by adding
these prediates as well, and the new abstraction is again checked for
correctness starting the cycle anew.

This process (though using different techniques to check the Boolean
programs) is the foundation of the SLAM tool from Microsoft
Research. This tool is able to check many properties of Windows device
drivers, and has since the release of Vista been a part of the
official Windows Driver Development Kit under the name Static Driver
Verifier (SDV). Getting a clean bill of health from the SDV is needed
for the Windows logo. 

Drivers are an attractive target for verification because they tend to
be relatively small, tend to not make heavy use of the heap, have many
well-defined properties to check, and are relatively important
(because bugs will bring down your whole system). From Microsoft's
perspective, drivers are an attractive target because if a third-party
driver blue-screens your system, they get the blame anyway.

-----------------


\section{Transition Systems}

We can formalize the type of structure pictured in Figure 2 as
something called a \emph{transition system}. The idea of a transition
system is really quite simple: it's a (possibly-infinite) collection
of states along with a \emph{transition relation} that describes how
states are related. Formally, a transition system is a pair $(S,
\rightarrow)$ where $S$ is some set and $\rightarrow \subseteq S
\times S$. (We will write $p \rightarrow q$ for $(p, q) \in
\rightarrow$.) It can be visualized as a possibly-infinite directed
graph.

When modeling a Boolean program $B$, we will treat the semantics of
the program as a transition system where the states $S$ are the set of
possible configurations of the program $B$ and the transition relation
$\rightarrow$ is any state transition that respects the semantics of
the program $B$.

--------------------------

An example of another transition system is defined by the
$\lambda$ calculus. States in this transition system correspond to
terms in the $\lambda$ calculus in De Bruijn form, and the transition
relation sets $s \rightarrow t$ exactly when $t$ can be obtained from
$s$ via a $\beta$ reduction. (We use De Bruijn form so we can ignore
$\alpha$ renamings, which don't change the ``meaning'' of a
$\lambda$ term. Of course, using ``normal'' $\lambda$ notation and
including $\alpha$ renamings in the transition relation, and/or
including $\eta$ reductions, would also result in a perfectly valid
transition system.)

The following diagram shows some portions of this transition system,
written in normal notation instead of De Bruijn for clarity:

\begin{verbatim}
  [ diagram of reductions of something diverging then converging
    and terminating, then (\x.xx)(\x.xx), then (\x.xxx)(\x.xxx) ]
\end{verbatim}

Starting from any term in this transition system can result in (1)
eventually reaching a normal form, (2) reaching a loop in the
transition system, or (3) reaching an infinite chain of transitions
that never reaches a normal form. All three cases are illustrated
above.

--------------------------

We are going to think of the semantics of a Boolean program as a
transition system, and essentially check reachability properties on
this pushdown system. However, the transition system can be infinite,
and so we need a finite representation. That is where pushdown systems
come into play. (Figure 2 illustrates an infinite transition system
borne out of a program. That program has the property that if
execution reaches the infinite portion, then it will never terminate,
but this does not need to be the case: it is possible to have a
program with an infinite number of reachable states but where from any
configuration, a halting state is reachable.)


\section{Pushdown Systems}

A pushdown system is a way of describing a transition system where:
\begin{enumerate}
\item Every state can be written \Config{p}{s} where $p$ is a
  ``control location'' drawn from a finite set $P$ and $s$ is a stack
  of items drawn from a finite set $\Gamma$. (The name ``control
  location'' is misleading here for our purposes: when modeling a
  Boolean program, we will use the control location to hold
  information about the current valuation of global variables.)
\item To decide what states $q$ in the transition system are
  successors of a given state $p$ (i.e. $p \rightarrow q$), you need
  only look at the top item in the $p$'s stack. The transition may
  push or pop items from the stack (as well as change the control
  location), though we restrict it from popping more than one item.
\end{enumerate}

If you are familiar with a pushdown automaton (PDA), a PDS is very
similar except there is no input tape or input charracter component of
the transitions. If you like, you can think of a PDS as exploring the
configuraton space of a PDA when you have complete control over the
input.

The two above restrictions means that we can define a PDS as follows:

\begin{definition} A pushdown system (PDS) $\mathcal{P}$ is a triple $(P, \Gamma,
  \Delta)$ where $P$ is a finite set (the ``control locations''),
  $\Gamma$ is also a finite set (the ``stack alphabet''), and $\Delta$
  is a set of rules of the form
  \Rule{\Config{p}{\gamma}}{\Config{q}{w}} where $p, q \in P$, $\gamma
  \in \Gamma$, and $w \in \Gamma^*$.
\end{definition}

A PDS defines a transition system in the following way. The set of
states in the transition system is the set of all configurations
\Config{p}{w} for $p \in P$ and $w \in \Gamma^*$. The transition
relation is defined as follows: $\Config{p}{\gamma\alpha} \rightarrow
\Config{q}{w\alpha}$ (with $\gamma \in \Gamma$ and $\alpha \in
\Gamma^*$) if and only if $\Rule{\Config{p}{\gamma}}{\Config{q}{w}}$
is a rule in the PDS.

Finally, to simplify the post* and pre* algorithms (presented later),
we add one more restriction: every PDS rule must replace the top item
on the stack with either zero, one, or two other elements. (This is
not a fundamental restriction: it is possible to simulate any PDS
without this restriction by one with it, simply by adding new
intermediate states.) Rules then either one of the following forms:
\begin{itemize}
\item \Rule{\Config{p}{\gamma}}{\Config{p'}{\epsilon}} (which is called a
$\Delta_0$ rule)
\item \Rule{\Config{p}{\gamma}}{\Config{p'}{\gamma'}} (a $\Delta_1$
  rule)
\item \Rule{\Config{p}{\gamma}}{\Config{p'}{\gamma'\gamma''}} (a
  $\Delta_2$ rule)
\end{itemize}


We will be interested in two kinds of reachability questions: forward
reachability, called post*, and backward reachability, called pre*. In
both cases, we ask what is reachable from a starting set of
configurations $C$, and the answer will be the set of reachable (or
reaching) configurations. The definitions are as follows:
\begin{itemize}
\item $post^*_{\mathcal{P}}(C) = \{p' | \exists p\in C. p \rightarrow^*
  p'\}$
\item $pre^*_{\mathcal{P}}(C) = \{p | \exists p'\in C. p \rightarrow^*
  p'\}$
\end{itemize}
In both cases, $\rightarrow^*$ is the reflexive transitive closure of
the transition relation defined by the PDS $\mathcal{P}$. I'll also
abuse notation and use $post^*(p)$ to mean $post^*(\{p\})$, and
similarly for pre*.

Reachability questions can then be expressed in two ways: if we want
to know whether a configuration $p'$ is reachable from a configuration
$p$, we can ask whether $p' \in post^*(p)$ or whether $p \in
pre^*(p')$. (In the case of modeling Boolean programs, $p$ will
typically be the starting configuration (or the set of starting
configurations) and $p'$ will be the error configuration (or set of
error configurations).

So by expressing a transition system as a PDS we have solved the
problem of having to finitely represent an infinite transition system,
but we are not done: the input and output of both pre* and post* may
themselves be infinite sets, and we need a way of representing those
sets in such a way that we can still compute pre* and post*.

\subsection{Representing Sets of Configurations}

In order to represent a set of configurations symbolically (and
finitely), we will use a slightly-modified form of finite
automaton. For a PDS $\mathcal{P}$, we will define a
$\mathcal{P}$-automaton as $(Q, \delta, F)$. (I'll usually drop the
$\mathcal{P}$ suffix and just refer to it as an automaton or
configuration automaton.) $Q$ is a set of states, and must be a
superset of $P$ (the PDS's set of states); $\delta: Q \times
\Gamma_\epsilon \rightarrow \powerset{Q}$ is the transition function;
$F$ is the set of accepting states. The automaton's alphabet (normally
$\Sigma$) is the same as the PDS's stack alphabet $\Gamma$, plus
$\epsilon$.

A configuration automaton $\mathcal{A}$ accepts a configuration
\Config{p}{w} if $\mathcal{A}$ starts in its state $p$, follows
transitions according to the next unread symbol in $w$ (starting from
the top of the stack), and can end in an accepting state $f \in
F$. (This is just like the standard definition of acceptance in an FA
except that the machine starts in the configuration's state instead of
one fixed start state $q_0$, and the symbols it reads are from the
configuration's stack, top-to-bottom.)

For instance, the following atuomaton accepts the configuration
\Config{p}{abc}:



$\mathcal{P}$-automata can, of course, represent either finite or
infinite sets of configurations of $\mathcal{P}$. However, not every
infinite set of configurations can be reprsented by a
$\mathcal{P}$-automaton, just like not every infinite set of normal
strings can be represented by a standard finite automaton. If a set of
configurations \emph{can} be represented, then we say that class is
``regular''.

\subsection{The post* Algorithm}

An interesting and important fact is that for any PDS, both $post*(C)$
and $pre*(C)$ are regular if C is regular. Furthermore, both can be
computed starting from a automaton representation of $C$ by a simple
process. We will discuss the process for post* in this section, and
pre* in the next. Both have a similar flavor; which is easier is a
matter of some debate.

Both processes can be explained by presenting one or more simple
\emph{saturation rules}. A saturation rule tells you to find a
transition in the automaton and a rule in the PDS which match on their
state and stack symbol, and then add one or more other transitions if
they are not already present. This will hopfeully become clear
momentarily. 

Conceptually, you just check to see if there are any saturation rules
that can be applied; if there are, then you apply them, but if not,
the process is complete and the automaton now holds the set of
configurations that is the post* of what you started with. In an
actual implementation, this process becomes guided by a worklist; we
will examine that after describing the saturation rules.


Let's look at our first post* saturation rule. Suppose our automaton
looks like this, in part:



The squiggly bit on the right represents possibly a bunch of states
and transitions, but it could also be none at all. (In the extreme
case, the node labeled $n$ could be the same as $p$ in which case the
$\gamma$ transition is a self loop, and $p$ could be accepting.) Call
$L(n)$ be the set of strings accepted by $\mathcal{A}$ starting from
state $n$.

What does this mean? This means that a configuration is accepted by
this machine if it takes the form \Config{p}{\gamma\alpha}, where
$\alpha$ some string in $L(n)$.

Now consider the PDS rule
\Rule{\Config{p}{\gamma}}{\Config{p'}{\gamma'}}. This rule means that,
in the transition system defined by $\mathcal{P}$, a successor of
\Config{p}{\gamma\alpha} is \Config{p'}{\gamma'\alpha}. This means
that \Config{p'}{\gamma'\alpha} must be accepted by the post*
automaton.

What can we do to make this happen?

The answer is remarkably simple: we just add a transition
$\Trans{p'}{\gamma}{n}$.

In other words, we have our first saturation rule (we will have to
revise these rules later however, because of the presence of
$\epsilon$ transitions):

   If there is a PDS rule
   \Rule{\Config{p}{\gamma}}{\Config{p'}{\gamma'}} and a transition
   \Trans{p}{\gamma}{n}, add a new transition
   \Trans{p'}{\gamma'}{n}. ($p$, $p'$, and $n$ need not be distinct.)

OK. So now what should we do about $\Delta_0$ and $\Delta_2$ rules?
Before we go further, you might want to try to predict the saturation
rules for these.

Let's look at $\Delta_0$ first. This is actually very similar to the
$\Delta_1$ case we just did:

Consider the PDS rule
\Rule{\Config{p}{\gamma}}{\Config{p'}{\epsilon}}. This rule means that,
in the transition system defined by $\mathcal{P}$, a successor of
\Config{p}{\gamma\alpha} is \Config{p'}{\alpha}, and the latter
configuration must be accepted by the post* automaton.

To make this happen, we just add a transition like last time except
this time we make in an $\epsilon$ transition:
$\Trans{p'}{\gamma}{n}$.

Formally, we have the following rule:

   If there is a PDS rule
   \Rule{\Config{p}{\gamma}}{\Config{p'}{\epsilon}} and a transition
   \Trans{p}{\gamma}{n}, add a new transition
   \Trans{p'}{\epsilon}{n}.

Finally, consider the $\Delta_2$ rule
\Rule{\Config{p}{\gamma}}{\Config{p'}{\gamma'\gamma''}}. This rule
means that, in the transition system defined by $\mathcal{P}$, a
successor of \Config{p}{\gamma\alpha} is
\Config{p'}{\gamma'\gamma''\alpha}. This means that
\Config{p'}{\gamma'\gamma''\alpha} must be accepted by the post*
automaton.

This one is a bit more complicated, because we have to add \emph{two}
transitions. But what should be the intermediate state? For reasons
which we will describe later, we add one new state for each
$(p',\gamma')$ pair; we will call this state $q_{p',\gamma'}$.

Thus the automaton picture looks as follows:

And we have the saturation rule

   If there is a PDS rule
   \Rule{\Config{p}{\gamma}}{\Config{p'}{\gamma'\gamma''}} and a
   transition \Trans{p}{\gamma}{n}, add new transitions
   \Trans{p'}{\gamma'}{q_{p',\gamma'}} and
   \Trans{q_{p',\gamma'}}{\gamma''}{n}, adding the state $q_{p',\gamma'}$
   if necessary.


\subsubsection{Correcting our post* saturation rules for $\epsilon$
  transitions}

Unfortunately, the saturation rules in the preceeding section are not
yet completely correct. To illustrate the problem, consider a PDS with
the following rules:
   p A -> p B C
   p B -> p eps
   p C -> p D

By applying each of these rules in order, we see that the
configuration \Config{p}{D} is in $post*(\Config{p}{A})$.

However, let's apply the saturation rules exactly as given above,
starting from an automaton that accepts just the configuration
\Config{p}{A}. (You may want to try this yourself before reading on!)

We start with

then apply the $\Delta_2$ saturation rule because we have a PDS rule
\Rule{\Config{p}{A}}{\Rule{p}{B C}} and a transition \Trans{p}{A}{f},
and get this:

then we apply the $\Delta_0$ saturation rule because we have a PDS
rule \Rule{\Config{p}{B}}{\Config{p}{\epsilon}} and a transition
\Trans{p}{B}{q_{p,B}}, and arrive at:

You might hope that we can now apply the $\Delta_1$ transition rule
with the PDS rule \Rule{\Config{p}{C}}{\Config{p}{D}}, but as written
it does not apply: there is no transition starting at $p$ with symbol
$C$.

Of course, the problem is that there's a \emph{path} that starts at
$p$ and begins by reading $C$, but that path starts with an $\epsilon$
transition.

(Note: a similar situation can arise with the $\epsilon$ transition on
the ``other side'' of the transition we're interested in, but only if
the initial query automaton has certain structures.)

We can handle this situation in two ways. The first is we modify our
three saturation rules so that they look for \emph{paths} rather than
just single transitions. If we use the notation \epspath{p}{\gamma}{n}
to mean a path with (optionally) some $\epsilon$ transitions, then a
$\gamma$ transition, then (optionally) some more $\epsilon$ transitions,
then our revised rules are:


   If there is a PDS rule
   \Rule{\Config{p}{\gamma}}{\Config{p'}{\gamma'}} and a path
   \epspath{p}{\gamma}{n}, add a new transition
   \Trans{p'}{\gamma'}{n}. ($p$, $p'$, and $n$ need not be distinct.)

   If there is a PDS rule
   \Rule{\Config{p}{\gamma}}{\Config{p'}{\epsilon}} and a path
   \epspath{p}{\gamma}{n}, add a new transition
   \Trans{p'}{\epsilon}{n}.

   If there is a PDS rule
   \Rule{\Config{p}{\gamma}}{\Config{p'}{\gamma'\gamma''}} and a path
   \epspath{p}{\gamma}{n}, add new transitions
   \Trans{p'}{\gamma'}{q_{p',\gamma'}} and
   \Trans{q_{p',\gamma'}}{\gamma''}{n}, adding the state
   $q_{p',\gamma'}$ if necessary.


The other option is to perform on-the-fly $\epsilon$ closure. We'll
leave the $\epsilon$ transitions in-place, but just ensure that we'll
never need to use them. To do this, we add one more rule:

    If there is a transition \Trans{q}{\epsilon}{q'} and
    \Trans{q'}{\sigma}{q''} (where $\sigma$ can either be a stack
    symbol in $\Gamma$ or $\epsilon$), add the transition
    \Trans{q}{\sigma}{q''}.


\subsubsection{The extra state in $\Delta_2$ rules.}

In this section we will attempt to explain the intuition behind the
behavior of the state added when dealing with $\Delta_2$ rules.  In
particular, we want to answer (1) why do we have to ``combine'' states
to begin with (we'll see what we mean by that), (2) why can we combine
the states that we do, and (3) why don't we combine more.

\subsubsubsection{Why do we have to combine states?}

Consider the PDS with the following rule:

  \Rule{\Config{p}{a}}{\Config{p}{ab}}

Starting from the configuration $c = \Config{p}{a}$, it is hopefully
clear that $post*(\{c\})$ should be the set of configurations of the
form \Config{p}{ab*}. (That is, the stack top is $a$ and under it is
any number of $b$s.

Let's see what happens if we use a different $\Delta_2$ saturation
rule -- one that at first glance seems like it could be reasonable,
and you might have thought of -- so that it will \emph{always} add a
new state. For simplicity's sake, I'll use the version that ignores
$\epsilon$ transitions, but of course we still need to handle that
somehow. Stated formally:

   If there is a PDS rule
   \Rule{\Config{p}{\gamma}}{\Config{p'}{\gamma'\gamma''}} and a
   transition \Trans{p}{\gamma}{n}, add a fresh state $q$ then add
   transitions \Trans{p'}{\gamma'}{q} and \Trans{q}{\gamma''}{n}.

Runnng post* with this rule gives the following. We start with the
query automaton that represents $\{c\}$:

Then apply the $\Delta_2$ saturation rule:

Then apply the $\Delta_2$ saturation rule again:

And again:

And again:

You see the problem.

If we want post* to terminate, we want to be able to avoid adding this
chain of fresh states. The approach we'll take is to \emph{combine}
all of these states -- combine states with a common source and
symbol. So we take all of the states in the dotted region and combine
it into one:


Note that the self-loop on $q_{p,a}$ is present because there are
transitions from one $q$ to another $q$ in the expanded
version. Furthermore, note that this is the same automaton we get if
we apply the correct $\Delta_2$ saturation rule:


\subsubsubsection{Why can we combine these states?}

Take the same PDS as before, but add a rule to it:

    \Rule{\Config{p}{a}}{\Config{p}{b}}

and then take our partially-done expanded post* automaton:

Now what can we do? Well, we can run our $\Delta_1$ saturation rule
and add the following transitions:


An interesting fact emerges. Every transition we added before to one
of our fresh states will prompt us to add the same \emph{additional}
transitions (just to each respective state). Call these transitions
$T$.  Furthermore, because we only ever add transitions starting from
a $p$ state (except when first creating a new fresh state), every
accepting path that traverses the fresh states at all will take one
transition in $T$ then follow the ``spine'' of fresh states to the
end. Finally, every transition along the spine will have the same
label because they were all added as the second transition when
looking at the same $\Delta_2$ rule.

But we can express that a different way: we can collapse all the fresh
states into one, which gives us the correct result.

\subsubsubsection{Why can't we combine more?}

You just can't.

I don't have as convincing an argument to make here; just it doesn't
make sense.


\subsection{The pre* algorithm}

Now that we've looked at post*, let's look at pre*. In many ways this
is simpler. Before continuing, you may want to try to produce the
saturation rule(s) yourself: remember that our goal is to produce an
automaton that expresses the set of configurations \emph{from} which a
configuration in the original query automaton is reachable. The
presentation below (in accordance with the literature) will present
just one saturation rule for any PDS rule
\Rule{\Config{p}{\gamma}}{\Config{p}{w}} ($w \in \Gamma^*$), but you
can either do that or just give separate saturation rules for $\Delta_0$,
$\Delta_1$, and $\Delta_2$ PDS rules as we did for post*.

Let's think through what we want in a similar way as before. Suppose
that we have a configuration $c = \Config{p}{w}$ that is
represented by our current automaton. We want to add some
backwards-reachable configurations. Well, how did we get to $c$? If we
have a rule \Rule{\Config{p'}{\gamma}}{\Config{p}{w'}} where $w'$ is a
prefix of $w$ (let $w = w'w''$), that gives us one possible way: we
start in the configuration \Config{p'}{\gamma w''}, apply that rule,
then arrive in $\Config{p'}{w'w''} = c$.

How does this look in the automaton world? If $c$ is accepted by the
current automaton, then there's a series of transitions out of $p$
labeled with each symbol in $w'w''$. To accept \Config{p'}{\gamma
  w''}, we just need to add a $\gamma$ transition from $p'$ to the
point at which we've skipped over $w'$. Graphically:



For $\Delta_0$ rules, we add a self-loop on a $p$ state.

That's all there is to it.


\section{From Unweighted to Weighted PDSs}

Above we described pushdown systems from the perspective of doing
model checking -- it is explicitly exploring the state space of a
program. Even when paired with something like SLAM, which is doing
iterative refinement of a ``real'' program, we are still doing
explicit-state model checking of the current Boolean program model.

Practically speaking it can be useful to think about WPDSs in a bit
different terms. While they \emph{are} a way of defining a weighted
transition system blah blah blah, what they are usually used for is
essentially dataflow analysis (abstract interpretation) over some
program. I'll try to describe things in those terms. In the final
section, I'll try to relate that picture to the somewhat more abstract
presentation in the literature.

Finally, one notion that I didn't get right away was what is meant by
\emph{weight}. What you should not have as a model in your mind is
something like a weighted graph from graph theory, like you might run
some weighted shortest path problem on. This picture has only enough
truth in it that I can't say it's \emph{completely} incorrect. Weights
in general are a much more abstract notion; as you'll see momentarily,
in our setting, they will represent abstract transformers.

\subsection{Weights for dataflow analysis and the MOAVP goals}

Consider some typical dataflow analysis. There's some domain $D$ of
possible dataflow facts (e.g. $\powerset(vars \times
\bbN_\bot^\top)$ for constant propagation and a ``meet'' ($\meet$)
that is used to combine dataflow facts when paths in the program
converge. (We are using a dataflow-analysis viewpoint: $\bot$
represents ``anything is possible'' and $\meet$ moves us closer to
$\bot$.) When propagating information ``through'' a basic block
$b$, you use some transformer $\tau_b: D \rightarrow D$ to translate
the dataflow fact before the block to the one after the block, or vice
versa (depending on whether you're doing a forwards or backwards
analysis). The goal is to find the ``meet over all valid paths''
solution from the program's start to its finish.

(If you're familiar with ``meet over all paths'' but not ``valid
paths'', a ``valid path'' is one where control does not use some call
site $c$ to call a function \Code{foo} but then return to a different
call site. This is what gives us context sensitivity.)

When doing dataflow analysis with a WPDS, we actually treat the set of
transformers $\tau_b$ explicitly. Our \emph{weight domain} then
consists of the set of these transformers and their compositions, and
some operations on them. We need two operations, and both are pretty
familiar.

The first opreation is composition. If we take some path through the
program where the transformers are $\tau_1$, $\tau_2$, and $\tau_3$,
then a transformer that represents the net effect of the path is just
$\tau_3 \circ \tau_2 \circ \tau_1$. Note, however, how using $\circ$
``reverses'' the path; i.e. the transformer that was used last appears
first when we write it out notationally. This is inconvenient, so we
invent a ``reversed composition'' operator. Since we have to make
something up anyway, we'll match the usual WPDS literature and use
$\tau_1 \extend \tau_2 \extend \tau_3$ for $\tau_3 \circ \tau_2 \circ
\tau_1$. We pronounce $\extend$ as ``extend'', but really it's just
function composition.

The second operation is meet, $\meet$. We know how to meet two data
values $d_1$ and $d_2$, but not necessarily transformers. But it's
pretty simple: $\tau_1 \meet \tau_2$ is going to be a new function
defined as follows: $(\tau_1 \meet \tau_2)(d) = (\tau_1(d)) \meet
(\tau_2(d))$. In other words, the meet of two functions takes its
argument, applies each of the two functions individually, then meets
the result.

This gives us the elements we'll need for our weight domain. 

When computing a forwards-flow analysis then, what our goal is as
follows:
    - For each program point p
        - paths = {all valid paths from start to p}
        - label p with meet(paths)
This labels each point $p$ with an aggrigate transformer that
represents all possible ways of getting to $p$. We can then apply that
transformer to the starting condition to obtain the dataflow fact that
holds at $p$. Forwards-flow problems will be computed with weighted
post*.

When computing a backwards flow analysis, the goal is similar:
    - For each program point p
        - paths = {all valid paths from p to finish}
        - label p with meet(paths)
This labels each point with an aggrigate transformer that represents
all possible ways of getting from that point to the end of the
program. Backwards-flow problems will be computed with weighted pre*.

Both of these problems generalize. We can start from any regular set
of configurations instead of just a single start node, or end in any
regular set of configurations. For instance, we could ask about
configurations where we specify not just a program point but also a
set of possible stacks: perhaps we are only interested in a point $p$
if \Code{foo} is somewhere on the stack.  Such \emph{stack-qualified
  queries} are a strict increase in power of WPDSs relative to other
methods for context-sensitive analysis, such as Shriri-Pnuelli's
framework (with their $\Phi$ functions).


\subsection{Creating a WPDS model of a program}

TODO


\subsection{Representing weighted configuration sets}

We will represent sets of weighted configurations in almost the same
way as before. However, instead of standard finite automata, we will
use a weighted FA (WFAs). Structurally, this just means that each
transition is annotated with a weight. (Later we'll add weights to
states as well.)

Unfortunately, the interpretation of these WFAs differs slightly
between pre* and post*.

\subsubsection{Interpreting the pre* automaton}
For pre*, the interpretation is still pretty simple. In the unweighted
case, a path from an initial state $p$ to an accepting state with
$\gamma_1\gamma_2\dots\gamma_n$ as the edge symbols indicates that the
configuration $c = \Config{p}{\gamma_1\gamma_2\dots\gamma_n}$ is
contained within the set in question. In the weighted case, it
indicates that $c$ is accepted \emph{with weight} $w_1 \extend w_2
\extend \dots \extend w_n$, where $w_i$ is the weight on the
transition that corresponds to $\gamma_i$.

Actually, the previous sentence is a lie. What it indicates is that
there is \emph{some} set of paths in the WPDS's transition system for
which the meet of the weights of those paths is $w_1 \extend w_2
\extend \dots \extend w_n$. However, it may be the case that the
automaton has \emph{multiple} paths that accept the same
configuration. (Don't get paths through the WPDS's transition system
confused with paths through the automaton!) In such a case, we need to
look at every such path, take its weight, then take the meet of all of
those.

\subsubsection{Interpreting the post* automaton}

Interpreting the weights of the post* automaton is a bit
different. Most things stay the same regarding what paths through the
WFA we look at and such, but computing the weight of a path is the
opposite. Here's the rule: the weight of a path $\gamma_1 \gamma_2
\dots \gamma_n$ is $w_n \extend \dots \extend w_2 \extend w_1$ (with
$w_i$ being the weight on the transition with symbol $\gamma_i$, as
before). In other words, it's the reverse of before\footnotemark.

\footnotetext{In a bit of irony this means we could revert to using
  $\circ$ instead of $\extend$, then write it in forward
  order. However, I'll stick with $\extend$ to match the existing
  literature.}

In the remainder of this section, I'll attempt to explain why.

You can kind of think of following a path in the pre* automaton in the
following way. If we want to know whether a WPDS is in an accepted
configuration, we successively pop off each stack symbol while taking
a corresponding transition.

-- something missing about why PA makes sense --

Interpretation of the post* automaton is a bit different. Because
post* and pre* are the opposite, we have to do the opposite thing:
Instead of running the WPDS then the WFA, we will do the opposite, and
run the WFA followed by the WPDS. We also think of the interpretation
of the WFA in the opposite: Instead of saying that a configuration is
accepted if we can pop it off using WFA transitions, we'll say it's
accepted if we can start with an empty stack and \emph{push} symbols
to get the stack in question (pushing a symbol for each WFA
transition).

But there's something funky here: a path with symbols $\gamma_1
\gamma_2 \dots \gamma_n$ will, if we just push the symbols in
sequence, give us the stack $\gamma_n \dots \gamma_2 \gamma_1$, with
$\gamma_n$ at the top of the stack. This isn't the interpretation we
meant: transitions out of the initial states should be the start
symbols. To get at the stack with $\gamma_1$ at the top, we need to
start at the other end -- in other words, we need to push the symbols
in the \emph{reverse order} of the path we accept them.

(Note that if we ignore weights, we haven't changed anything from the
unweighted post* case! If we want to know if a configuration is
accepted regardless of its weight\footnotemark, we still follow
transitions starting from the top of the stack, same as before. The
reversal of the path argument has to deal with how the weights are
built up only.)

The fact that we conceptually pushed $\gamma_n$ first means that we've
arrived at our rule: the \emph{weight} of a path $\gamma_1 \gamma_2
\dots \gamma_n$ is actually $\gamma_n \extend \dots \extend \gamma_2
\extend \gamma_1$.





\footnotetext{One caveat: if a configuration has the weight $\top$
  it's reasonable to say that is unreachable. If no transitions in the
  query automaton and no WPDS rules have weight $\top$ however, this
  will never arise.}


\subsection{Putting it all together}

What we are eventually interested in is the set of dataflow
values that can arise at a program point $n$. This corresponds to
multiple possible configurations -- and we want the meet of the weight
of each of those configurations. This leads to the following, naive,
algorithm for doing forwards-flow dataflow analyses:
    - Set up a query automaton from the initial configuration(s)
    - Run post*
    - For each prgoram point $n$
          - $w_{all} = \top$
          - For \emph{every path} $p$ that starts with $n$ from an
            initial state to an accepting state
              - Let $w$ be the weight of the path (using $\extend$),
                remembering that we are extending in reverse
              - $w_{all} = w_{all} \meet w$
          - Label $n$ with $w_{all}$

Later we'll look at getting rid of that explicit ``look at all
paths.''

The corresponding algorithm backwards-flow analysis is similar, but
this is a complication. There can be configurations in the pre*
automaton that are not actually reachable from the initial state. For
instance, if we run pre* on the example from section blah (from the
program's exit state with no stack), we get the following automaton:

This automaton accepts configurations like \Config{p}{foo_3 main_3}
where $main_3$ isn't a call site!

This configuration doesn't make sense as far as the program's
semantics are concerned, but the WPDS's transition system doesn't know
that. As far as the transition system is concerned, if it starts in
the configuration \Config{p}{foo_3 main_3}, then it can get to the
final state -- and indeed this is the case. 

So what we want to do is simple conceptually. We will run an
\emph{unweighted} post* from the set of initial configurations, then
intersect the result with our WFA from pre*. (WFA intersection
proceeds as normal FA intersection, except that weights are computed
in the resulting automaton by ???.) We then examine *this* automaton.

  The reason this didn't come up in the unweighted case is that we
  were asking a different question. At first glance we'd still have
  the same problem, since the unweighted pre* automaton would still
  accept that configuration. However, in the unweighted case we'd just
  ask whether a particular configuration (or any of a set of
  configurations) is accepted. Assuming our configuration of interest
  isn't one of these infeasible configurations, their presence doesn't
  matter.
  
  The problem comes from the fact that now we want to get the dataflow
  fact for some node $n$, and we do it by looking at all of the paths
  in the WFA that start with $n$. But now the presence of infeasible
  paths matters, because we want to exclude them.

So here is the process of carring out a backwards dataflow problem:
\begin{verbatim}
    - C_end = query automaton for exit configuration(s)
    - a_pre = pre(C_end)
    - C_start = query automaton for starting configuration(s)
    - a_post = post(C_start)
    - answer = a_pre \cap a_post
    - For each prgoram point $n$
          - $w_{all} = \top$
          - For \emph{every path} $p$ in answer that starts with $n$
            from an initial state to an accepting state
              - Let $w$ be the weight of the path (using $\extend$,
                remembering we are extending ``forward'')
              - $w_{all} = w_{all} \meet w$
          - Label $n$ with $w_{all}$
\end{verbatim}

\subsection{Weighted post*}

Finally, we have to discuss how to carry out the post* and pre*
algorithms. We'll start with post*.

The general idea of both algorithms is very similar to the unweighted
case. Really the only difference is we will change each rule from
something of the form ``look for a PDS rule and FA transition blah
blah and add a new transition blah blah'' and turn it into ``look for
blah blah and \emph{update the weight on transition} blah blah''
(instead of just adding it). We run the saturation rules until not
only are there no new transitions that are necessary but until no rule
will change any weight.

So let's look at the $\Delta_1$ case. The unweighted rule was

   If there is a PDS rule
   \Rule{\Config{p}{\gamma}}{\Config{p'}{\gamma'}} and a transition
   \Trans{p}{\gamma}{n}, add a new transition
   \Trans{p'}{\gamma'}{n}. ($p$, $p'$, and $n$ need not be distinct.)

Let's look at what it means in the weighted case. Suppose that the
weight of the PDS rule is $w_r$, the weight of the transition
\Trans{p}{\gamma}{n} is $w_t$, and the weight of the transition
\Trans{p'}{\gamma'}{n} is currently $w$. If the latter transition does
not yet exist, then we'll set $w = \top$. (Remember, $\top \meet x =
x$. It essentially reflects an impossible weight.) Conceptually, you
can think of \emph{every} transition being initially present in the
automaton but with weight $\top$, though you have to revise some other
explanations in light of this.

Here's a diagram of the transition system and the WFA:

Suppose we have some path $\pi$ in the WFA that starts at $n$ and
leads to an accepting state, and that path has weight $w_\pi$. (And
for sake of simplicity of the explanation, suppose there is only one
path with $\pi$'s labels.)

// start eding here invalid markup
This means that it is possible to start in the transition system from some
  configuration in the query set and arrive at the configuration
  \Config{p}{\gamma \pi} while the MOAVP weight is $w_t \extend
  w_\pi$.\footnotemark
If we Similarly, the MOAVPs solution from the query set to
  \Config{p'}{\gamma'\pi} is $w_r \extend w_t \extend w_\pi$


\footnotetext{Really it's the meet of all MOAVPs solutions from
  anything in the query set: ....}

Here's a diagram of the transition system and the WFA:

If we suppose that this is the first time we're looking at this
$\Delta_1$ rule, there are at least two sets of possibilities for how
to get to the configuration \Config{p'}{\gamma'\pi}. Because
$\gamma'\pi$ was already accepted, whatever paths we found before is
one such set. The second way is what we just discovered: use one of
the ways to get to the configuration \Config{p}{\gamma \pi}, then the
PDS $\Delta_1$ rule we are looking at. Here is a picture:

We want the MOAVPs solution, so we need to take the meet of all of
these paths. The weight of the first set of paths is $w_{t'} \extend
w_\pi$.\footnotemark The weight of the second set of paths is $w_t
\extend w_\pi$ to get to \Config{p}{\gamma \pi} then the weiht of the
rule, giving $w_r \extend w_t \extend w_\pi$.

\footnotetext{Here and in the next moment we are using a fact which I
  haven't really told you, which is that $a \extend (b \meet c) = (a
  \extend b) \meet (a \extend c)$. This doesn't necessarily always
  hold -- but it must for WPDSs to work.}

This means that the new weight for \Config{p'}{\gamma' \pi} should be
$(w_{t'} \extend w_\pi) \meet (w_r \extend w_t \extend w_\pi)$ which
equals $(w_{t'} \meet (w_r \extend w_t)) \extend w_\pi$.

Note that we can achieve this (at least for this path) by updating
$w_{t'}$ to be $w_{t'} \meet (w_r \extend w_t)$. Furthermore, you can
repeat this argument for any other paths and configurations that would
be affected by this change, and find out that the same weight will
work out for each of them.

This gives our saturation rule:

   If there is a PDS rule
   \Rule{\Config{p}{\gamma}}{\Config{p'}{\gamma'}} with weight $w_r$,
   a transition \Trans{p}{\gamma}{n} with weight $w_t$, and a
   transition \Trans{p'}{\gamma'}{n} with weight $w_{t'}$, update the
   weight on the transition according to $w_{t'}' = w_{t'} \meet (w_r
   \extend w_t)$. If \Trans{p'}{\gamma'}{n} is not present, add it
   with weight $w_t \extend w_r$.

(Note that the case where it was not present will result in the same
   effect as treating it as being there but with weight $\top$. I
   won't include that final clause in the future.)

And here is a graphical picture of what it means:


We'll come back to $\Delta_0$ rules, but here's the $\Delta_2$
saturation rule:

   If there is a PDS rule
   \Rule{\Config{p}{\gamma}}{\Config{p'}{\gamma'\gamma''}} with weight
   $w_r$, a transition \Trans{p}{\gamma}{n} with weight $w_t$, and a
   transition \Trans{p'}{\gamma'}{q_{p',\gamma'}} with weight
   $w_{t'}$, then update $w_{t'}$ following $w_{t'}' = w_{t'} \meet
   (w_r \extend w_t)$. Finally, add the transition
   \Trans{q_{p',\gamma'}}{\gamma''}{n} with weight $\lambda x.x$ if it
   does not already exist.

Note that the transition outgoing from the state $q_{p',\gamma'}$ will
\emph{always} be $\lambda x. x$, as we only ever modify the weight of
transitions out of an initial state. (Though without seeing the
$\Delta_0$ saturation rule you won't know that for sure, of course.)

Here's a diagram:


Finally we have the rule for $\Delta_0$. We are going to weave
$\epsilon$ closure into this rule, so it is a bit more complicated
than the other rules we've seen.

The first part of the rule is what you might expect:

   If there is a PDS rule
   \Rule{\Config{p}{\gamma}}{\Config{p'}{\epsilon}} with weight $w_r$,
   a transition \Trans{p}{\gamma}{n} with weight $w_t$, and a
   transition \Trans{p'}{\epsilon}{n} with weight $w_{t'}$, update the
   weight on the transition according to $w_{t'}' = w_{t'} \meet (w_r
   \extend w_t)$.

And now we need to perform $\epsilon$ closure. The unweighted version
would continue:

   Now for any transition \Trans{n}{\sigma}{n'}, add a transition
   \Trans{p'}{\sigma}{n'}.

(Note that $\sigma$ may be $\epsilon$, in which case adding
\Trans{p'}{\sigma}{n'} will also give you more things to do with
this rule later.)

So what should the weight be on \Trans{p'}{\sigma}{n'}? Well, for
starters, if that transition is already present then it will already
have a weight $w_{t''}$, so we'll have to meet in that. (Whenever we
update a weight we always take the meet with the old version.) But
what's the new information we've discovered?

   Now for any transition \Trans{n}{\sigma}{n'} with weight $w_{t''}$,
   update the weight on the transition \Trans{p'}{\sigma}{n'}
   according to 





   If there is a PDS rule
   \Rule{\Config{p}{\gamma}}{\Config{p'}{\epsilon}} with weight $w_r$,
   a transition \Trans{p}{\gamma}{n} with weight $w_t$, and a
   transition \Trans{p'}{\epsilon}{n} with weight $w_{t'}$, ...



\end{document}
